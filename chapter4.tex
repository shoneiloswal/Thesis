\documentclass[main.tex]{subfiles}
%
\begin{document}
%
\chapter[Parameter estimation]{Parameter estimation}
%
\section[Variables of interest]{Variables of interest}
Discuss the possible variables that can be extracted.
Which variables for which system parameter
%
\section[Statistical data]{Statistical data}
Present and discuss the statistical data of the a.m. parameters
Explain the problem with amplitude and pulse duration
%
\section[Scaling]{Scaling}
Why scaling is used and how was it implemented
MAY help eliminate amplitude deviations
%
\section[Variably peaked signal front]{Variably peaked signal front}
Occurence much more often. 
Affecting slope calculations
%
\section[Earthing using metal knob]{Earthing using metal knob}
Ampltidue really low.
Multiple peaks still there.
No advantage. 
Determination of slope seems infeasible AS OF NOW.
Estimate and control slug length.
%
\section[Shortening of input channel length]{Shortening of input channel length}
Too many unknown effects.
Long length also not good for control. excessive dead time.
Cut short length.
Show statistics and reason for the choice made.
%
\section[Statistics of delpeak]{Statistics of delpeak}
Present model and validation data
%
\section[Curve fitting]{Curve fitting}
Possiblities and alternative chosen. Why?
Parameter Estimation complete.
Show how you implemented it in LV.
\end{document}