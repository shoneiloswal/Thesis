\documentclass[main.tex]{subfiles}
%
\begin{document}
\chapter[Introduction]{Introduction}
Micro process engineering is a revolutionary technology of conducting chemical and physical processes in small volumina, typically in channels with an inner diameter in the sub-millimeter range. It offers high potential for significant improvement of process efficiency over conventional continuous flow systems. Due to small dimensions exceeding upto a maximum of 1~mm, this technology allows enhanced heat and mass transfer. For example, the diffusion length in micro-channels is comparable to its width, thus facilitating mixing of components in time of the order of micro-seconds. Similarly, excellent heat transfer capabilities allow conducting highly exothermic under isothermal conditions. Such characteristics can be well exploited for process intensification.

A key research focus in micro process engineering is the behaviour of two-phase flow in micro-channels. Biphasic liquid-liquid slug flows can be operated with high perfomances because of their inherently special hydrodynamics  and therefore, form the subject of this study. The special properties of multiphase flow systems as given by \todo{ref Jovanovic et al.} are attributed to three key differences from conventional macro flow systems. 

Formation of slugs of the order of channel width result in high specific surface areas which, inturn, leads to intensified mass transfer as mentioned above. \todo{ref Kashid et al.} reported a two order increase in magnitude of mass transfer compared to conventional methods. Internal circulation and eddys inside the slugs additionally increase the interfacial mass transfer. 

Secondly, the interfacial tension is not negligible compared to the viscous forces. Consequently, it is possible to create and regulate multiphase flows in the form of slugs under a given set of system conditions. Thus, a consistent flow regime can be established as previously discussed by several authors \todo{ref 4,7,8}. This is called the slug flow regime, which forms a core part of the considerations of this work.

Finally, the developed flow is strongly influenced by the roughness and the wetting properties of the wall material. These influences can be exploited for guiding the flow. \todo{ref Scheiff et al. 9} has shown that differences in the hydrophilicity of the wall material can be used for phase separation via a side channel. 
\todo{Add paper info sent by Prof. Agar}

\end{document}