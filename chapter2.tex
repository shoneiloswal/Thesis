\documentclass[main.tex]{subfiles}
%
\begin{document}
\chapter[Implementaton of online measurement]{Implementation of online measurement}
%
In order to implement a control structure, the signals from the electrode need to be read and processed online. This chapter deals with the initial obstacles faced in doing so. Firstly, the software interface was changed mainly to facilitate online signal processing and subsequently, control. Secondly, the read signals are digitally processed to increase the Signal to Noise(SNR) ratio. This aids in reading and extracting the signal characteristics, such as amplitude, slope and period, better. Finally, a electrode of fixed dimensions and thus, reproducible in nature was identified. 
%
\section[Initial Scenario]{Initial Scenario}
Discuss the intial state of the project. what were the ideas? What problems were there? The new ideas...
%
\section[Switch to LabVIEW]{Switch to LabVIEW}
Earlier, the voltage signals were read using a Digital Storage Oscilloscope(DSO) and its corresponding software program. The voltage data was recorded in MS-Excel. This data was then read into MATLAB and a spline regression was performed to extract the underlying signal. \todo{add figure of dso and spline}The major drawback of this process was that it is offline and thus, a continuous control on the liquid-liquid slug flow is not possible. Additonally, the software does not feature any means to digitally process or analyse the signal in any way. Conversely, a new software interface is required which can fulfill all the above mentioned limitations of the DSO and its allied software.
%
LABVIEW facilitates not only reading the signals online, but also digitally process them, for example filtering, analyse and subsequently control desired variables. A NI-USB 6212 measuring card has been used as an interface between the plant and LABVIEW program. 	The voltage output from the amplifier is fed to the measuring card. This voltage is continuously read into LABVIEW using its in-built Data Acquisition System(DAQ) as a digital signal. The measuring card has a maximum sampling frequency of 400k~Samples/s.  \todo{insert Labview noisy signal} is an example of a signal sampled at 1000~samples/s in LABVIEW.
%
\section[Signal filtering]{Signal filtering}
test
\section[Reproducible electrode]{Reproducible electrode}
Until now, the electrode being used was formed by dropping molten tin into a Polytetrafluoroethylene(PTFE) mould with the capillary in-situ. \todo{figure of mould with capillary in-situ} Molten tin engulfs the capillary in the form of a ring thus forming the electrode. Often, the molten tin solidified quickly before it could completely surround the capillary and form the ring. This method yields electrodes of slightly different dimensions each time \todo{electrode with different dimensions}. Thus, in general, to achieve higher reproducibility and eliminate any added influences, a search for an electrode with fixed dimensions was carried out. 

Based on the phenomenon of triboelectroc effect and induced polarisation, the following characteristics of an ideal electrode are enumerated:
\begin{enumerate}
	\item It is a very good electrical conductor and thus, can be polarised to a higher extent
	\item It should wrapped around the capillary as closely as possible, again, to achieve higher polarisation
	\item The lateral width of the electrode should be small enough such that voltage signals from consecutive slugs do not interfere each other
\end{enumerate}

Washers, nuts or simply electrical wires such as that of copper are the possible alternatives that fit the above mentioned criteria. 
In earlier studies, use of a copper wire as an electrode was studied. \todo{ref gatberg}. The wire was wound around the capillary as tight as possible and the voltage signals were recorded. Although, care had to be taken to avoid snapping the wire under the application of excess force. The results produced are shown in \todo{add figure from Gatberg}. The signal showed minor fluctuations which may be due to the fact that the copper wire cannot be wound perfectly tight. 

Washers and nuts of different sizes were other alternatives of an electrode. Steel and brass washers as well as nuts of different sizes were tested as an electrode. The following figures show the signals corresponding to each of the electrode respectively. \todo{add figure of washers and nuts}

It is evident from the above figures that the nuts produced better signals which can be digitally processed to extract its characteristics than do the washers. This is primarily because the washers do not fit around the capillary as do the nuts. Both, brass as well as steel nuts, produce voltage signals that can be used to extract signal characteristics. But, the brass nut is chosen owing to its higher electrical conductivity. 
\end{document}